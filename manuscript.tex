% ********************************************
%
%   Dear Authors,
% 
%   Please, read the following comments in Preset settings and adjust the settings according to your needs.
%   Please, feel free to add more packages or macros if needed.
%   Details of the pre-defined packages, symbols and macros can be found in the submodule "template". (See PDF files.)
%   If you finished the adjustments, you can remove the comments for a clean document.
% 
%   Any questions or contributions are welcome.
%   Please, contact to the author.
%   Enjoy your writing.
%
%                               Myeongseok Ryu
%  	    				dding_98@gm.gist.ac.kr
%                                  09.Feb.2025
%
% ********************************************

% ********************************************
% This project is highly ispired by the work of LMRES Lab, Hochschule München.
% Thank you very much my dear friend, Niklas Monzen, for your kind support.
% ********************************************

% ============================================
%         Preset 1. Document Type
% ============================================
\documentclass[conference]{IEEEtran}
% \documentclass[lettersize,journal]{IEEEtran}

% THIS IS NEEDED FOR FINAL SUBMISSION
\IEEEoverridecommandlockouts 
% This command is only needed if you want to use the \thanks commands  
%\overrideIEEEmargins 
% Needed to meet printer requirements.
%---------------------------------------------------------------------------- 

% ============================================
%         Preset 2. Local Path
% ============================================
% When you create "figures" or "movies" directories in the "src" (source code) directory, the file name including path can be long.
% To avoid this, and to make it easier to adjust, you can define the alias for the path.
% The examples are provided below.

\newcommand*{\FIGURESPATH}{./figures}
% \newcommand*{\SIMFIGURESPATH}{./src/script_simulation/figures}
% \newcommand*{\SLXFIGURESPATH}{./src/simulink_simulation/figures}
% \newcommand*{\MOVIESPATH}{./movies}

% ============================================
%         Preset 3. Pre-defined Settings
% ============================================
% PLEASE DO NOT ADD OR REMOVE PACKAGES IN THE SUBMODULE LOCALLY!
% CONTACT THE AUTHOR FOR ADJUSTMENTS.
%
% The packages are pre-defined in the submodule "Template".
% If you need more packages, please add them after using pre-defined packages.

\def\pub{false} % true for publication, false for draft
\newcommand*{\template}{template}
\input{\template/preamble/preamble_conf.tex}

% ============================================
%     Preset 4. Additional Corrections
% ============================================
% correct bad hyphenation here
% \hyphenation{op-tical net-works semi-conduc-tor}
% \pagestyle{empty}

% ============================================
%               TITLE and AUTHORS
% ============================================
\newcommand{\tit}{Example Title of Conference Template}
\begin{document}
\title{
	\tit
    % Example Title of Template
% {\footnotesize \textsuperscript{*}Note: Sub-titles are not captured for https://ieeexplore.ieee.org  and
% should not be used}
    \thanks{
        This work was supported by my 10 fingers and Macbook Air M1.
    }
}

\author{
    \IEEEauthorblockN{1\textsuperscript{st} Myeongseok Ryu}
    \IEEEauthorblockA{\textit{School of Mechanical and Robotics Engineering} \\
    \textit{Gwangju Institute of Science and Technology}\\
    Gwangju, Republic of Korea \\
    dding\_98@gm.gist.ac.kr}
\and
    \IEEEauthorblockN{2\textsuperscript{nd} Niklas Monzen}
    \IEEEauthorblockA{\textit{Laboratory for Mechatronic and Renewable Energy Systems (LMRES)} \\
    \textit{Hochschule München (HM) University of Applied Sciences}\\
    Munich, Germany \\
    niklas.monzen@hm.edu}
}

\maketitle 
\thispagestyle{empty}

% ============================================
%         ABSTRACT and KEYWORDS
% ============================================
\begin{abstract}
	This project is built to provide a template for writing papers and simulation.
\end{abstract}

\begin{IEEEkeywords}
	Scuderia Ferrari, Apple, Joan Gilbert, Gibson, Fender, Hugo BOSS
\end{IEEEkeywords}


\section*{Notation}
In this study, the following notation is used:

\begin{itemize}
    \item $\otimes$ denotes the Kronecker product \cite[Definition 7.1.2]{Bernstein:2009aa}.
    \item $\mv x=[x_i]_{i\in[1,\cdots,n]}\in\R^n$ and $
        \mm A
        := 
        [a_{ij}]
        _{
            i\in[1,\cdots,n],j\in[1,\cdots ,]
        }\allowbreak\in\R^{n\times m}
        $ denotes a vector and a matrix.
    \item $\myrow_i(\mm A)$ denotes the $i\textsuperscript{th}$ row of the matrix $\mm A\in\R^{n\times m}$. 
    \item $\myvec(\mm A):= [\myrow_1(\mm A^\top)  ,\cdots,\myrow_m(\mm A^\top)  ]^\top   $ for $\mm A\in\R^{n\times m}$.

    \item $\lambda_\text{min}(\mm A)$ denotes the minimum eigenvalue of the matrix $\mm A\in\R^{n\times n}$.
    \item $\mm I_n$ denotes the $n\times n$ identity matrix and $\mm 0_{n\times m}$ denotes the $n\times m$ zero matrix.
\end{itemize}

% ============================================
%         SECTION: Introduction
% ============================================
\section{Introduction}

This template is designed to provide a consistent format for writing papers and simulation reports.

% ============================================
%         SECTION: Example
% ============================================
\section{Example Section}

% ============================================
%            SECTION: Conclusion
% ============================================
\section{Conclusion}

% ============================================
%         BIBLIOGRAPHY
% ============================================
\bibliographystyle{ieeetr}
\bibliography{\template/refs}

\end{document}

